\documentclass[10pt,twocolumn]{witseiepaper}

\usepackage{KJN}

\ifpdf
\pdfinfo{
/Title (ELEN4012 - Feature Based Automatic Modulation Classification)
/Author (Jacques Visser and Anthony Farquharson)
}
\fi

\begin{document}

\title{ELEN4012 - Feature Based Automatic Modulation Classification}

\author{Jacques Visser and Anthony Farquharson
\thanks{School of Electrical \& Information Engineering, University of the
Witwatersrand, Private Bag 3, 2050, Johannesburg, South Africa}
}

% TODO Rewrite abstract once the rest of the contents are figured out.
\abstract{automatic modulation classification involves identifying the modulation scheme used in a signal without the decision being guided by an operator. This report covers a preliminary investigation into the design and implementation of such a system. An overview of the relevant literature is presented and proposals are made regarding the details of the implementation of such a system using and Ettus USRP.}

\keywords{modulation, classification, USRP, UHD}

\maketitle
\thispagestyle{empty}\pagestyle{empty}

% TODO How does one even write an introduction?
\section{INTRODUCTION}

\section{LITERATURE SURVEY}
There are three major recognized approaches to automatic modulation classification, as detailled by \cite{zhu2014automatic}.

\subsection{Feature Based Automatic Modulation Classification}
Feature based AMR has been shown to be non-ideal, but significantly less computationally intensive\cite{zhu2014automatic} than the aforementioned methods.

\section{CONSTRAINTS}

\section{PROPOSED DESIGN OVERVIEW}

\section{IMPLEMENTATION}
\subsection{USRP}
\subsection{UHD API}
\subsection{Build System}
\subsection{Classifier}

\section{TESTING}
\subsection{Simulated Testing}
\subsection{Practical Testing}


\bibliographystyle{witseie}
\bibliography{prelim} \end{document}
